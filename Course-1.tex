\documentclass[10pt,a4paper]{article}
% \usepackage[utf8]{inputenc}
\usepackage[latin5]{inputenc}
\usepackage[T1]{fontenc}
\usepackage{amsmath}
\usepackage{amssymb}
\usepackage{graphicx}
\usepackage[top=2.00cm]{geometry}
\usepackage[turkish]{babel}
%\usepackage{lipsum}
\author{Yasin Tosun}
\title{CoursEcon 2nd International Summer Seminar Program}
\date{August 15, 2022}

\begin{document}
	%1. "maketitle" and "section" are used to create automatic titles.
	
	\maketitle
	
	\section{Login}
	
	%2. We can comment this way. You cannot see these on the page.
	3%. "textbf" is used to make the texts bold.
	
	\textbf{Lesson 1}
	
	The main goals of countries are to make their economic growth sustainable. %4. Here you only need to leave a space to separate the words. Even if you put many spaces in between, it will be accepted as a single space.
	
	For this, they need to use the resources they have correctly and efficiently.
	%5.Paragraph adjustment is done automatically. Extended texts are automatically placed at the bottom.
	%6. To create a new paragraph, you must leave 1 blank line in between.
	
	After the Second World War, the efforts of states to gain superiority over each other changed to the principle of human capital and training qualified individuals.
	%7. The use of quotation marks varies. First "AlT gr + ;" you must do. Then use normal double quotes (Shift + 2).
	The theme of ``Tomorrow'' is our main idea this year.
	
	%8. If you want to make bold font in the text, use the code below.
	This year the congress theme was determined as \textbf{``Tomorrow''}.
	
	%9. Two different methods are used to italicize within the text and to italicize the text. First we will see the "textit" codes and then the "emph" codes.
	
	\textit{This year the congress theme was determined as \textbf{``Tomorrow''}.}
	
	%10. However, here you cannot use inverse matrix logic to convert a text in italics to plain. This doesn't work.
	
	\textit{This year \textit{congress} theme was determined as \textbf{``Tomorrow''}.}
	
	%11th. "emph" has the same functions as "textit", plus inverse matrix logic.
	
	\emph{This year the congress theme was determined as \textbf{``Tomorrow''}.}
	
	\emph{This year the congress theme was determined as \emph{\textbf{``Tomorrow''}}.}
	
	%12. If you want to underline, use "underline".
	
	\emph{This year the congress theme was determined as \underline{\emph{\textbf{``Tomorrow''}}}.}
	\newline
	%13. This code goes directly to the next line. Line spacing is adjusted automatically.
	
	\maketitle
	
	\section{Development}
	
	\textbf{Lesson 2}\newline
	
	%\lipsum[1]
	
	%14. "Installing the lipsum package.
	%15. The quote medium is used to quote another author or to line up important sentences. In this environment, text is arranged in a narrower space:
	
	\begin{quote}
		After the Second World War, the efforts of states to gain superiority over each other changed to the principle of human capital and training qualified individuals. After the Second World War, the efforts of states to gain superiority over each other changed to the principle of human capital and training qualified individuals.
		%18. To quote it, that is, to simply add an author, the "hfill" command is used.
		
		\hfill Solow
		
		%16. This acts as an internal quote. It allows writing at narrower intervals by leaving space at the top and bottom.
		
		%19. Sometimes you may need to use special letters. For example, the letter � can be written as \'{e}. Also, \^{a} can be written inside the hatted a.
		\hfill Ahmet T\^{a}nriverdi
		
	\end{quote}
	%17. If a line space is left after this text, Lesson 2 text below will become the beginning of the paragraph. However, if it does not leave a space, it will be treated as a sentence continuation and will be placed at the beginning of the line.
	\textbf{Lesson 2}
	
	\textbf{Lesson 2}
	
	
	\maketitle
	
	\section{Result}
	
	
	\begin{flushleft}
		Keyword
	\end{flushleft}
	%20. One of the sections we will use when preparing a summary is that you can write left-justified and spaced.
	
	\begin{flushright}
		Keyword
	\end{flushright}
	
	\maketitle
	
	\subsection{Lesson Over! break}
	%21. You can use "subsection" to create a subtext title.
	
\end{document}