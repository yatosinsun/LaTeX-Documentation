\documentclass[10pt,a4paper]{article}
\usepackage[latin5]{inputenc}
\usepackage[T1]{fontenc}
\usepackage{amsmath}
\usepackage{amssymb}
\usepackage{graphicx}
\usepackage[top=2.00cm]{geometry}
\usepackage[turkish]{babel}
\usepackage{lipsum}
\usepackage{ulem}

\begin{document}
	LESSON 3
	
	Let's start the listing process.
	
	\begin{itemize}
		%1. It allows listing to be done with punctuation marks.
		\item Article 1
		\begin{itemize}
			\item Article 1.1.
			\begin{itemize}
				\item Article 1.1.1.
			\end{itemize}
		\end{itemize}
	\end{itemize}
	
	\begin{enumerate}
		%2. It allows listing as enumeration.
		\item Article 1
		\item Article 2
		\begin{enumerate}
			%3. The second listing within the numbering is made with letters as (a).
			\item Article 2.1.
			\begin{enumerate}
				%4. The 3rd listing within the numbering is done with roman numerals in the form of i.
				\item Article 2.1.1.
				\begin{itemize}
					%5. Numbering can also be done with punctuation marks.
					\item Additional information
					\begin{itemize}
						%6. Punctuation can also be made within punctuation. The shapes of these punctuations vary.
						\item Additional information 1.1.
					\end{itemize}
				\end{itemize}
			\end{enumerate}
		\end{enumerate}
	\end{enumerate}
	
	Articles are in bold font, underlined, etc. can be done.
	
	\begin{enumerate}
		\item Questions
		\begin{enumerate}
			%7. Changes can be made with [text] commands.
			\item[\text{a.}] Answer 1
			\item[\textbf{a.}] Answer 2
			\item[\textit{a.}] Answer 3
			\item[\underline{a.}] Answer 4
			
			\item[\text{b)}] Answer 1
			\item[\textbf{b)}] Answer 2
			\item[\textit{b)}] Answer 3
			\item[\underline{b)}] Answer 4
			
		\end{enumerate}
	\end{enumerate}
	
	\begin{description}
		%8. Instead of automatic assignment, we can also explain it ourselves.
		\item[Option 1]
		\item[a.] Details
		\item[b.] Details
	\end{description}
	
	Let's try a new combination.
	
	\begin{enumerate}
		%9. Description logic can be made in [] parentheses.
		\item[\underline{\textbf{Option 1}}]
		\item[a.]
		\item[b.]
		\item[\underline{\textbf{Option 2}}]
	\end{enumerate}
	
	\begin{center}
		
		
		%10. To make a headline
		\textbf YOUNG ECONOMISTS CLUB
		
		%11th. There are 3 different fonts.
		%Serif Font Family \textrm
		%Sans Serif Font Family \textsf
		%Monospaced Font Family \texttt
		
		%12. For font changes
		%\textit for italics
		\textsl for %Slanted
		%\textsc for small caps
		%Slanted is a non-cursive version of Italic.
		
		\textsc{Young Economists Club}
		
		%13. Instead of point size, font size can be adjusted with codes such as \tiny, \small... below.
		
		{\tiny\textbf{YOUNG ECONOMISTS CLUB}}
		
		{\scriptsize\textbf{YOUNG ECONOMISTS CLUB}}
		
		{\footnotesize\textbf{YOUNG ECONOMISTS CLUB}}
		
		{\small\textbf{YOUNG ECONOMISTS CLUB}}
		
		{\normalsize\textbf{YOUNG ECONOMISTS CLUB}}
		
		{\large\textbf{YOUNG ECONOMISTS CLUB}}
		
		{\Large\textbf{YOUNG ECONOMISTS CLUB}}
		
		{\LARGE\textbf{YOUNG ECONOMISTS CLUB}}
		
		{\huge\textbf{YOUNG ECONOMISTS CLUB}}
		
		{\Huge\textbf{YOUNG ECONOMISTS CLUB}}
		
		%14. For typewriter font
		\texttt{YOUNG ECONOMISTS CLUB}
		
		{\huge\textbf\texttt{YOUNG ECONOMISTS CLUB}}
		
		%15. For plain font
		
		\textsf{Young Economists Club}
		
		{\huge\textbf\texttt{Young Economists Club}}
		
		16%. For Serif Font Family
		\textrm{Young Economists Club}
		
		{\huge\textbf{\textrm{Young Economists Club}}}
		
		%17. For slanted text [different from italic]
		
		\textsl{Young Economists Club}
		
		{\huge\textbf{\textsl{Young Economists Club}}}
		
		
		%18. To write underlined text;
		
		\uuline{\textbf{{YOUNG ECONOMISTS CLUB}}}
		
		\uline{YOUNG ECONOMISTS CLUB}
		%Underlined
		
		\uuline{YOUNG ECONOMISTS CLUB}
		% Double Underline
		
		\uwave{YOUNG ECONOMISTS CLUB}
		%Wavy Underline
		
		\sout{YOUNG ECONOMISTS CLUB}
		%Strike-through
		
		\xout{YOUNG ECONOMISTS CLUB}
		
		\dashuline{YOUNG ECONOMISTS CLUB}
		%Dash
		
		\dotuline{YOUNG ECONOMISTS CLUB}
		%Dotted Line
		
		\dotuline { }
		
		%To write old style numbers
		
		\oldstylenums{0123456789}
		
	\end{center}
	
	%19. To write letters of the foreign alphabet
	\'{to}
	
	\`{e}
	
	\^{e}
	
	\"{to}
	
	\~{e}
	
	\={e}
	
	\.{to}
	
	\v{e}

	\u{e}

	\H{e}

\end{document}