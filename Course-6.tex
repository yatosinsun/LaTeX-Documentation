\documentclass[10pt,a4paper]{article}
\usepackage[latin5]{inputenc}
\usepackage[T1]{fontenc}
\usepackage{amsmath}
\usepackage{amssymb}
\usepackage{graphicx}
\usepackage[top=2.00cm]{geometry}
\usepackage[turkish]{babel}
\usepackage{lipsum}
\usepackage{ulem}
\usepackage{setspace}
\usepackage{fancyhdr}
\usepackage{xcolor}
%It will enable the use of color.
\usepackage{rotating}
%This is the package required to rotate tables.
\usepackage{wrapfig}
%tables are used to be nested with text.
\usepackage{colortbl}
%Allows the loading of color palettes to create color tables.

\begin{document}
	
\renewcommand{\tablename}{\textbf{Table}}
%3. We can change the table names with this code.

\onehalfspacing

%1. To prepare a table in matrix form, run the basic code below.

\begin{center}
	
	\begin{tabular}{|c|c|c|}
		
		\multicolumn{3}{c}{\textbf{Table 1.1. Unit Root Test}} \\[1ex]
		
		%Here the first parenthesis contains the number of columns to be joined.
		%The second parenthesis represents the average.
		%The last parenthesis asks us to write the title.
		%one. You can add spaces under the text with the code [0.5ex]. "ex"; It is a more precise measurement than "cm".
		
		\hline
		\textbf{Department} & \textbf{Area} & \textbf{Class} \\ \cline{1-3}
		\hline
		Data1 & Data2 & Data5 \\
		\hline
		Veri3 & Veri4 & Veri6 \\ \hline
	\end{tabular}
	
	\vspace{0.15cm}
	
	\textbf{\textit{Source} :} \textit{OECD, 2020}
	
	
\end{center}

\lipsum[1]

\begin{table}[h]
	
	%h: here , t: top - top of the next page, b: Bottom - bottom of the page, p: page means another page.
	%We can customize the tables with this code. In particular, we can automatically sort the tables and create memory for referencing.
	%We tell LaTeX where to put the table with the codes %h - t - b.
	% Here "!" Putting the command somehow provides the "put this table here" command.
	% [h!] gives the command to place this here somehow. If he can do it, he will do it, if he can't, don't force him, he has signals too, bro :)
	
	\centering
	%Allows overall centering of the table.
	
	\begin{tabular}{|c|c|c|}
		
		\multicolumn{3}{c}{\textbf{Table 1.1. Unit Root Test}} \\[1ex]
		
		%Here the first parenthesis contains the number of columns to be joined.
		%The second parenthesis represents the average.
		%The last parenthesis asks us to write the title.
		%one. You can add spaces under the text with the code [0.5ex]. "ex"; It is a more precise measurement than "cm".
		
		\hline
		\textbf{Department} & \textbf{Area} & \textbf{Class} \\ \cline{1-3}
		\hline
		Data1 & Data2 & Data5 \\
		\hline
		Veri3 & Veri4 & Veri6 \\ \hline
		
	\end{tabular}
	\caption{Write note for Painting here.}
	\label{Table1}
	
\end{table}

\lipsum[2-3]

\begin{table}[h!]
	
	\centering
	
	\caption{Write note for Painting here.}
	\label{Table2}
	% Here we define it as a reference in the command we give inside the label.
	%When we want to refer to these tables in the text, it will be seen if you call it with the name in curly brackets.
	
	\begin{tabular}{|c|c|c|}
		
		\multicolumn{3}{c}{\textbf{Table 1.1. Unit Root Test}} \\[1ex]
		
		%Here the first parenthesis contains the number of columns to be joined.
		%The second parenthesis represents the average.
		%The last parenthesis asks us to write the title.
		%one. You can add spaces under the text with the code [0.5ex]. "ex"; It is a more precise measurement than "cm".
		
		\hline
		\textbf{Department} & \textbf{Area} & \textbf{Class} \\ \cline{1-3}
		\hline
		Data1 & Data2 & Data5 \\
		\hline
		Veri3 & Veri4 & Veri6 \\ \hline
		
	\end{tabular}
	
	
\end{table}

%Now let's imagine that we refer to these tables here. Let's first examine it as Table \ref{Table1} and then Table \ref{Table2}.

% The label command we have given makes it easier for the \ref command to work.

%Texts can also be colored with % \textcolor{color}{text}.

%Now let's imagine that we refer to these tables here. Let's first examine it as \textcolor{red}{Table \ref{Table1}} and then \textcolor{red}{Table \ref{Table2}}.
%You can search "LaTeX:Using Color" for more colors.
	
\begin{sideways}
	
	%You can make the tables horizontal with this code.
	
	\begin{tabular}{|c|c|c|}
		
		\multicolumn{3}{c}{\textbf{Table 1.1. Unit Root Test}} \\[1ex]
		
		%Here the first parenthesis contains the number of columns to be joined.
		%The second parenthesis represents the average.
		%The last parenthesis asks us to write the title.
		%one. You can add spaces under the text with the code [0.5ex]. "ex"; It is a more precise measurement than "cm".
		
		\hline
		\textbf{Department} & \textbf{Area} & \textbf{Class} \\ \cline{1-3}
		\hline
		Data1 & Data2 & Data5 \\
		\hline
		Veri3 & Veri4 & Veri6 \\ \hline
		
	\end{tabular}
	
	
\end{sideways}


\begin{rotate}{45}
	
	%You can place tables at an angle with this code.
	
	\begin{tabular}{|c|c|c|}
		
		\multicolumn{3}{c}{\textbf{Table 1.1. Unit Root Test}} \\[1ex]
		
		%Here the first parenthesis contains the number of columns to be joined.
		%The second parenthesis represents the average.
		%The last parenthesis asks us to write the title.
		%one. You can add spaces under the text with the code [0.5ex]. "ex"; It is a more precise measurement than "cm".
		
		\hline
		\textbf{Department} & \textbf{Area} & \textbf{Class} \\ \cline{1-3}
		\hline
		Data1 & Data2 & Data5 \\
		\hline
		Veri3 & Veri4 & Veri6 \\ \hline
		
	\end{tabular}
	
\end{rotate}

\vspace{5cm}

%To place it within the text, it must first be placed between two texts. We add lipsum codes to the beginning and end.

\lipsum[1]

\begin{wraptable}{r}{5cm}
	
	%You can make the tables horizontal with this code.
	
	\begin{tabular}{|c|c|c|}
		
		\multicolumn{3}{c}{\textbf{Table 1.1. Unit Root Test}} \\[1ex]
		
		%Here the first parenthesis contains the number of columns to be joined.
		%The second parenthesis represents the average.
		%The last parenthesis asks us to write the title.
		%one. You can add spaces under the text with the code [0.5ex]. "ex"; It is a more precise measurement than "cm".
		
		\hline
		\textbf{Department} & \textbf{Area} & \textbf{Class} \\ \cline{1-3}
		\hline
		Data1 & Data2 & Data5 \\
		\hline
		Veri3 & Veri4 & Veri6 \\ \hline
		
	\end{tabular}
	
\end{wraptable}

\newpage

\lipsum[2-4]

\vspace{0.3cm}

\begin{tabular}{|c|c|c|}
	
	\rowcolor[rgb]{0,0,1}
	
	%rgb: red-green-blue
	%Numbers are opacity. It lies between 0 and 1. 1 is exactly 0 and 0 is no visibility.
	
	\multicolumn{3}{c}{\textcolor{white}{\textbf{Table 1.1. Unit Root Test}}} \\[1ex]
	\hline
	%Here the first parenthesis contains the number of columns to be joined.
	%The second parenthesis represents the average.
	%The last parenthesis asks us to write the title.
	%one. You can add spaces under the text with the code [0.5ex]. "ex"; It is a more precise measurement than "cm".
	\rowcolor[gray]{0.6} \textbf{Department} & \textbf{Area} & \textbf{Class} \\ \cline{1-3}
	\hline
	\rowcolor[gray]{0.8} Data1 & Data2 & Data5 \\
	\hline
	\rowcolor[gray]{0.9} Veri3 & Veri4 & Veri6 \\
	\hline
	Color can be assigned to rows with %rowcolor. Here, approaching 0 is darkness and 1 is lightness.
	
	\end{tabular}

	\vspace{0.3cm}

	\lipsum[2]
	
	
\end{document}